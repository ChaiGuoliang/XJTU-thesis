% !TeX root = ../main.tex

\xchapter{算法与代码}{Algorithm and Code}

\begin{tcolorbox}[colback=red!5!white,colframe=red!75!black]
  学校并未对 算法 或 代码 的排版格式作出限定,因此可以自行修改。
\end{tcolorbox}

\xsection{算法}{Algorithm}

考虑到实际使用需求,即长算法分段和注释功能,模板不再使用 \clist[columbiablue]{algorithm2e} 作为默认算法排版工具,改为调用 \clist[columbiablue]{algpseudocode}。注意,这两个宏包相互冲突。算法的具体编写方式请自行查阅宏包,即\clist{texdoc algpseudocode}。

\begin{texcode}[]{}
\begin{algorithm}[H]
  \caption{How to write algorithms}
  \begin{algorithmic}[1]
      \State \textbf{Input:} this text
      \State \textbf{Output:} how to write algorithm with \LaTeX2e \Comment{Just a comment}
      \State initialization
      \While{not at end of this document}
        \State read current
        \If{understand}
          \State go to next section
          \State current section becomes this one
        \Else{$k\geq k^\text{max}$}
          \State go back to the beginning of current section
        \EndIf
      \EndWhile
  \end{algorithmic}
\end{algorithm}
\end{texcode}

\begin{texcode}[]{}
  \begin{algorithm}[H]
    \caption{Simulation-optimization heuristic}\label{algorithm:SOH}
    \begin{algorithmic}[1]
        \State \textbf{Data:} current period $t$, initial inventory $I_{t-1}$, initial capital $B_{t-1}$, demand samples
        \State \textbf{Output:} Optimal order quantity $Q^{\ast}_{t}$
        \State $r\leftarrow t$
        \State $\Delta B^{\ast}\leftarrow -\infty$
        \While{$\Delta B\leq \Delta B^{\ast}$ and $r\leq T$}
          \State $Q\leftarrow\arg\max_{Q\geq 0}\Delta B^{Q}_{t,r}(I_{t-1},B_{t-1})$
          \State $\Delta B\leftarrow \Delta B^{Q}_{t,r}(I_{t-1},B_{t-1})/(r-t+1)$
          \If{$\Delta B\geq \Delta B^{\ast}$}
            \State $Q^{\ast}\leftarrow Q$
            \State $\Delta B^{\ast}\leftarrow \Delta B$
          \EndIf
          \State $r\leftarrow r+1$
        \EndWhile
    \end{algorithmic}
  \end{algorithm}
\end{texcode}



\xsection{导入代码}{Input codes}

从现有文件导入源码,使用 \clist[columbiablue]{lstlisting} 宏包,本模板预先定义了一基本格式 \clist{sty_basic} 以供使用,可以自定义新的样式。

\begin{texcodeonly}
\lstinputlisting[style=sty_basic,
  language=matlab,
  caption={标准粒子群算法},
  label={listing:a_label}]{./Codes/pso.m}
\end{texcodeonly}
\lstinputlisting[style=sty_basic,
                    language=matlab,
                    caption={标准粒子群算法},
                    label={listing:a_label}]{./Codes/pso.m}