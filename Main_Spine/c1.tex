% !TeX root = ../main.tex

\xchapter{绪论}{Introductions}

\xsection{背景}{Backgrounds}

本文档仅提供部分可能用到的示例,系统的学习 \LaTeX 用法请参考其他书目,如 \emph{lshort},可通过在 shell(命令行/cmd) 中执行 \clist{texdoc lshort-zh} 获得中文版。

同时,务必意识到 \LaTeX 的工作及使用方式与 Microsoft Word 有极大的不同,不要问「怎么在 \LaTeX 中实现 Word 的 xx 功能」。

\xsection{符号说明}{Examples}

为了更加清楚的说明本模板的用法,特意使用了如下几种标记方式来标记「\LaTeX 源码」及「说明」:

\clist{行内的灰色字块}表示简短的代码(\LaTeX 或 shell);

\clist[columbiablue]{行内的蓝色字块}表示宏包的名称。

\begin{tcolorbox}
  灰色框用来放置有一定长度的 \LaTeX 源码(及执行结果)
\end{tcolorbox}

\begin{tcolorbox}[colback=red!5!white,colframe=red!75!black]
  红色框用来放置学校的学位论文模板的\textbf{格式}要求
\end{tcolorbox}

\begin{tcolorbox}[colback=blue!5!white,colframe=blue!75!black]
  蓝色框用来放置学校的学位论文模板的\textbf{内容}要求
\end{tcolorbox}

\begin{tcolorbox}[colback=blue!5!white,colframe=blue!75!black,title=绪论部分的要求]
  绪论部分主要论述论文的选题意义及应用背景、国内外研究现状分析及论文的主要研究内容等。
\end{tcolorbox}

\clearpage

\xsection{基本功能}{Basic Functions}

\xsubsection{文档层级}{Level of this document}

按照要求,正文最多应具有七个层级。其中,为了方便中英目录的生成,模板重新定义了前三级标题的命令,并新增了第六级的命令。注意,不论正文主体语言是否为英文,标题顺序均为先中后英(前三级)。

\begin{table}[H]
  \begin{tblr}{
    colspec={X[c]X[c]},
    verb
  }
    \toprule
      层级 & 命令 \\
    \midrule
      0  & \verb|\xchapter{Chs}{Eng}| \\
      1  & \verb|\xsection{Chs}{Eng}| \\
      2  & \verb|\xsubsection{Chs}{Eng}| \\
      3  & \verb|\subsubsection{Chs/Eng}| \\
      4  & \verb|\paragraph{Chs/Eng}| \\
      5  & \verb|\subparagraph{Chs/Eng}| \\
      6  & \verb|\subsubparagraph{Chs/Eng}| \\
    \bottomrule
  \end{tblr}
\end{table}

示例如下:

\subsubsection{第四级标题$1$}

\paragraph{第五级标题$1$}

\paragraph{第五级标题$2$}

\paragraph{第五级标题$3$}

\subparagraph{第六级标题$1$}

\subparagraph{第六级标题$2$}

\subparagraph{第六级标题$3$}

第六级下的一些文字

\subsubparagraph{第七级标题$1$}

第七级下的一些文字

\clearpage

\xsubsection{加粗,斜体与其他字体设置}{Bold, Emph and Other Font Settings}

在中文学位论文的写作中,不推荐使用加粗,斜体等方式突出重点。

使用\clist{\textbf{text}}进行可以对字加粗;同时,请注意到在英文中,正确的强调方式不是使用加粗,而是使用斜体,即使用\clist{\emph{text}}进行强调。如:
\begin{texcode}[]{}
  为了突出\textbf{重点},英文可以这样:This is the \emph{key question}.
\end{texcode}

使用\clist{\textit{text}}、\clist{\underline{text}}等实现斜体,下划线等格式。





\xsubsection{脚注及其使用}{Footnotes}

\begin{tcolorbox}[colback=red!5!white,colframe=red!75!black]
  脚注可用小号字(一般小五号宋体)列在相应正文同一页最下部并与正文部分用细线(版面宽度的1/4长)隔开。
\end{tcolorbox}

脚注\footnote{脚注序号“\ding{172},……,\ding{180}”的字体是“正文”,不是“上标”,序号与脚注内容文字之间空$1$个半角字符,脚注的段落格式为:单倍行距,段前空$0$磅,段后空$0$磅,悬挂缩进$1.5$字符;中文用宋体,字号为小五号,英文和数字用Times New Roman字体,字号为$9$磅;中英文混排时,所有标点符号(例如逗号“,”、括号“()”等)一律使用中文输入状态下的标点符号,但小数点采用英文状态下的样式“.”。}\footnote{脚注序号“\ding{172},……,\ding{180}”的字体是“正文”,不是“上标”,序号与脚注内容文字之间空$1$个半角字符,脚注的段落格式为:单倍行距,段前空$0$磅,段后空$0$磅,悬挂缩进$1.5$字符;中文用宋体,字号为小五号,英文和数字用Times New Roman字体,字号为$9$磅;中英文混排时,所有标点符号(例如逗号“,”、括号“()”等)一律使用中文输入状态下的标点符号,但小数点采用英文状态下的样式“.”。}\footnote{脚注序号“\ding{172},……,\ding{180}”的字体是“正文”,不是“上标”,序号与脚注内容文字之间空$1$个半角字符,脚注的段落格式为:单倍行距,段前空$0$磅,段后空$0$磅,悬挂缩进$1.5$字符;中文用宋体,字号为小五号,英文和数字用Times New Roman字体,字号为$9$磅;中英文混排时,所有标点符号(例如逗号“,”、括号“()”等)一律使用中文输入状态下的标点符号,但小数点采用英文状态下的样式“.”。}是对文中有关内容的解释、说明或补充,使用上角标(序号\ding{172}、\ding{173}、$\cdots$)进行标注。

脚注可以通过 \clist{\footnote{}}自动编号生成,也可使用\clist{\footnotetex{text}}手动添加脚注。

摘要中关于项目资助的标识,是使用脚注生成的:

\begin{tcolorbox}
  \lstinline|\footnotetext{*本研究得到某某基金(编号:)的资助}|
\end{tcolorbox}


\clearpage

\xsection{数字、公式和定理环境}{Equation and Theorem}

\xsubsection{数字与单位}{Numbers and Units}

模板引入了\clist[columbiablue]{siunitx}宏包实现数字与单位的正确排版,具体使用细节请自行查看该宏包,即 \clist{texdoc siunitx}。

为了输入不同格式及角度的数字,可以使用\clist{\num{3.45d-4}}得到\num{3.45d-4},使用\clist{\num{-e10}}得到\num{-e10},使用\clist{\ang{1;2;3}}得到\ang{1;2;3},避免了手动调节格式的问题。

该宏包预先定义了部分单位,可以直接调用,同时生成较为复杂的单位。如:

使用\clist{\SI{10}{\hertz}}得到 \SI{10}{\hertz}。

\xsubsection{公式、矩阵与数学符号}{Equations, Matrix and Mathematical Symbols}


\begin{tcolorbox}[colback=red!5!white,colframe=red!75!black]
  \begin{enumerate}[leftmargin=0.5cm]
    \item 公式应另起一行,居中编排,较长的公式尽可能在等号后换行,或者在“+”、“-”等符号后换行。公式中分数线的横线,长短要分清,主要的横线应与等号取平。
    \item 公式后应注明编号,公式号应置于小括号中,如公式(2-3)。写在右边行末,中间不加虚线;
    \item 公式下面的“式中:”两字左起顶格编排,后接符号及其解释;解释顺序为先左后右,先上后下;解释与解释之间用“;”隔开。
    \item 公式中各物理量及量纲均按国际标准(SI)及国家规定的法定符号和法定计量单位标注,禁止使用已废弃的符号和计量单位。
  \end{enumerate}
\end{tcolorbox}

此部分是 \LaTeX 中比较有趣的一部分,具体的使用细节请自行查阅资料。

\begin{texcode}[]{}
  公式如下:
  \begin{equation}
      -e^{\max}_\text{dis} \leq a_t \leq e^{\max}_\text{ch}\label{equation:c1:mdl:cstr_dis}
  \end{equation}
  式中:$e$为xxx;$a$为xxx。
\end{texcode}

所以如\cref{equation:c1:mdl:cstr_dis}所示:
\begin{equation}\label{equation:c1:exp}
    -e^{\max}_\text{dis} \leq a_t \leq e^{\max}_\text{ch}
\end{equation}

其中,数学公式可以使用\clist{\boldmath..\unboldmath}对整体进行加粗:

\begin{texcode}[]{}
\boldmath
\begin{equation}
  \biggl(\int_{-\infty}^\infty e^{-x^2}\diff x\biggr)^2 
\end{equation}
\unboldmath
\end{texcode}

如果只需对某个元素调整,可以通过\clist{\boldsymbol},或使用\clist[columbiablue]{unicode-math}提供的,以\clist{\sym}开始的命令族,如\clist{\symup, \symbfup, \symit, \symbfit}:
\begin{texcode}[]{}
  \begin{equation}
    \biggl(\int_{-\infty}^\infty \symbfit{e}^{-\symup{x}^2}\diff \symbfup{x}\biggr)^2 
  \end{equation}
\end{texcode}
不建议再使用标准 \LaTeX 及 amsmath 扩展提供的 \clist{\mathbf}、\clist{\mathcal} 等输入不同样式的字母,可能会导致使用文本字体而非数学字体(的粗体形式)。

请注意,本模板已经定义了\clist{\diff}命令作为微分符号,请不要自己使用新的符号,使用效果如下:
\begin{align}\label{equation:c1:diff}
  \biggl(\int_{-\infty}^\infty e^{-x^2}\diff x\biggr)^2 
    &= \int_{-\infty}^\infty\int_{-\infty}^\infty e^{-(x^2+y^2)}\diff x\diff y \\
    &= \int_0^{2\pi}\int_0^\infty e^{-r^2}r\diff r\diff\theta \\
    &= \int_0^{2\pi}\biggl(-{e^{-r^2}\over2}\bigg\vert_{r=0}^{r=\infty}\,\biggr)\diff\theta \\
    &= \pi
\end{align}


同时,定义了部分用于简化输入的命令:

\begin{texcode}[sidebyside]{}
  $\seq{x}{n}$ \\ % serise
  $\iprod{i}{j}$ \\ % inner product
  $\mbs{B}$  % \boldsymbol
\end{texcode}

\xsubsection{定理相关}{Theorems}\label{section:Environments for math}

本模板通过\clist[columbiablue]{thmtools}宏包定义了以下环境,其中\textbf{证明}环境默认具有缩进,没有编号,同时使用\clist{\qedhere}结束证明,其他环境样式一致。

\begin{table}[H]
  \begin{tblr}{
    colspec={*{6}{X[c]}},
    vline{4},
  }
  \toprule
    环境 & 英文名称 & 中文名称 & 环境 & 英文名称 & 中文名称 \\
  \midrule
    proof & Proof & 证明 & theorem & Theorem & 定理 \\
    axiom & Axiom & 公理 & corollary & Corollary & 推论 \\
    lemma & Lemma & 引理 & definition & Definition & 定义 \\
    example & Example & 例子 & proposition & Proposition & 命题 \\
    assumption & Assumption & 假设 & remark & Remark & 注 \\
    problem & Problem & 问题 & conjecture & Conjecture & 猜想 \\
  \bottomrule
  \end{tblr}
\end{table}


\begin{texcode}[]{}
  \begin{theorem}[勾股定理]\label{theorem:pt}
    若 $a,b$ 为直角三角形的两条直角边,$c$ 为斜边,那么 $a^2 + b^2 + c^2.$
  \end{theorem}

  \begin{proof}
    通过\ldots

    所以:
    \begin{equation*}
        G(x, y) = G(y, x).  \qedhere
    \end{equation*}
  \end{proof}
\end{texcode}

\begin{proposition}
  所以:
  \begin{equation*}
      G(x, y) = G(y, x).
  \end{equation*}
\end{proposition}

\begin{conjecture}[勾股定理]
    若 $a,b$ 为直角三角形的两条直角边,$c$ 为斜边,那么 $a^2 + b^2 + c^2.$
\end{conjecture}

\begin{axiom}[勾股定理]
    若 $a,b$ 为直角三角形的两条直角边,$c$ 为斜边,那么 $a^2 + b^2 + c^2.$ 若 $a,b$ 为直角三角形的两条直角边,$c$ 为斜边,那么 $a^2 + b^2 + c^2.$ 若 $a,b$ 为直角三角形的两条直角边,$c$ 为斜边,那么 $a^2 + b^2 + c^2.$
\end{axiom}

\begin{definition}[勾股定理]
    若 $a,b$ 为直角三角形的两条直角边,$c$ 为斜边,那么 $a^2 + b^2 + c^2.$
\end{definition}

\xsection{其他环境}{Other Environments}

\xsubsection{枚举环境}{Enumerates}

\begin{texcode}[]{}
  \begin{enumerate}
    \item 123
    \item 231421
    \item 124124
  \end{enumerate}
\end{texcode}


\xsection{正文}{Text}

\zhlipsum[1]

\xsubsection{正文}{Text}

\zhlipsum[2]

\xsubsection{正文}{Text}

\zhlipsum[3]